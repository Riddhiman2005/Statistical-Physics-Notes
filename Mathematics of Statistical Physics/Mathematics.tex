
\documentclass{article}

% Language setting
% Replace `english' with e.g. `spanish' to change the document language
\usepackage[english]{babel}

% Set page size and margins
% Replace `letterpaper' with `a4paper' for UK/EU standard size
\usepackage[letterpaper,top=2cm,bottom=2cm,left=3cm,right=3cm,marginparwidth=1.75cm]{geometry}

% Useful packages
\usepackage{amsmath}
\usepackage{graphicx}
\usepackage[colorlinks=true, allcolors=blue]{hyperref}

\title{\textbf{Some Mathematics Required In Statistical Physics}}
\author{Riddhiman Bhattacharya}

\begin{document}
\maketitle


\section{Probablity}
At the heart of statistical physics lies the concept of probability. Instead of focusing
on the precise details of individual particles, statistical physics examines the statistical
distribution of their properties. This statistical approach allows us to make predictions
about the macroscopic behavior of a system based on its microscopic constituents. 

\subsection{One Random Variable}
\subsubsection{Probability density function (PDF)}

$$ p(x) = \text{prob}(E \subset [x, x + dx])$$


\subsubsection{Cumulative probability function (CPF)}

$$ P(x) = \text{prob}(E \subset [-\infty, x])$$

$$ p(x) \equiv \frac{dP(x)}{dx} $$


\subsubsection{Expectation value}

$$\langle F(x) \rangle = \int_{-\infty}^{+\infty} dx \, p(x) \, F(x) $$

$$p_F(f) \, df = \sum_i p(x_i) \, dx_i
$$
$$\Longrightarrow
 p_F(f) = \sum_i p(x_i) \left| \frac{dx}{dF} \right|_{x=x_i} $$



\subsubsection{Moments of the PDF}

$$m_n \equiv \langle x^n \rangle = \int dx \, p(x) \, x^n $$

\subsection{Many Random Variables}
\subsubsection{Change of Variables}

$$\frac{dy_1 \cdots dy_n}{dx_1 \cdots dx_n} = \left| \frac{\partial (y_1, \ldots, y_n)}{\partial (x_1, \ldots, x_n)} \right|$$

Here the substitution function is injective and continuously differentiable, and the differentials transform as in above equation.

$$|\mathbf{J}_{ij}| = \left| \frac{\partial y_i}{\partial x_j} \right|
$$
$$\int_{\varphi(U)} f(v) \,dv = f(\varphi(u)) \left| \frac{\partial v_i}{\partial u_j} \right| \, du $$




\subsubsection{Central Limit Theorem}

If are $n$ random samples drawn from a population with overall mean $\mu $ and finite variance $\sigma^2$ , and if $\overline{X}_n
$
is the sample mean, then the limiting form of the distribution is the  standard normal distribution.

$$Z = \lim_{n \to \infty} \sqrt{n} \left( \overline{X}_n - \frac{\mu}{\sigma} \right) = \frac{1}{\sqrt{2\pi}} e^{-\frac{1}{2}X^2}
$$

\subsection{Large Numbers}

\subsubsection{Saddle Point Integration}
$$\tau = \int dx \, e^{N\phi(x)} 
 $$

 $$e^{N\phi(x_{\text{max}})} \int dx \exp\left[\frac{N}{2}|\phi''(x_{\text{max}})|(x-x_{\text{max}})^2\right] \approx \sqrt{\frac{2\pi}{N|\phi''(x_{\text{max}})|}} \, e^{N\phi(x_{\text{max}})}
$$


\subsubsection{Stirling's Approximation}
Let $\phi(x) = \ln x - \frac{x}{N}
$ in the Saddle point integration, so we get

$$\ln N! = N \ln N - N + \ln(2\pi N) + \frac{1}{2} \ln(2\pi N) + O\left(\frac{1}{N}\right)$$




\section{Some Important Theorems and Results}
\subsection{Euler's Homogenous Function Theorem}
\textbf{\Large Statement:} \textit{\large If $f$ is a partial function of real variables that is positively homogeneous of degree $k$ , and continuously differentiable in some open subset of $ \mathbb{R}^n $,then it satisfies in this open set the partial differential equation}

$$\sum_{i=1}^{n} x_i \left(\frac{\partial f}{\partial x_i}\right)(x_1, \ldots, x_n) = k f(x_1, \ldots, x_n)$$


\subsection{Important Results}
\begin{itemize}
    \item $\int_{-\infty}^{\infty} e^{-a(x+b)^2} \, dx = \sqrt{\frac{\pi}{a}}$
    \item Let $f(x,y,z)=0 $ and let $w$ be function 2 of them
    \begin{itemize}
        \item $$\left(\frac{\partial x}{\partial y}\right)_z = \frac{1}{\left(\frac{\partial y}{\partial x}\right)_z} $$
        \item $$\left(\frac{\partial x}{\partial y}\right)_w \cdot \left(\frac{\partial y}{\partial z}\right)_w = \left(\frac{\partial x}{\partial z}\right)_w$$
        \item $$\left(\frac{\partial x}{\partial y}\right)_z \cdot \left(\frac{\partial y}{\partial z}\right)_x \cdot \left(\frac{\partial z}{\partial x}\right)_y = -1$$
        
    \end{itemize}

\end{itemize}

\end{document}
